%%%%%%%%%%%%%%%%%%%% author.tex %%%%%%%%%%%%%%%%%%%%%%%%%%%%%%%%%%%
%
% template for chapters to the Handbook of Exoplanets
% modified by H. Deeg from the 'template.tex' provided by Springer for the svmult.cls class
% 20Mar 2016
%
%%%%%%%%%%%%%%%% Springer %%%%%%%%%%%%%%%%%%%%%%%%%%%%%%%%%%


% RECOMMENDED %%%%%%%%%%%%%%%%%%%%%%%%%%%%%%%%%%%%%%%%%%%%%%%%%%%
\documentclass[graybox,natbib,nosecnum]{svmult}
\bibpunct{(}{)}{;}{a}{}{,} % suppress commas between author-names and year

\pdfoutput=1   %forces use of pdflatex. Disable if you prefer to use .eps or .ps figures.
% choose options for [] as required from the list
% in the Reference Guide

\usepackage{mathptmx}       % selects Times Roman as basic font
\usepackage{helvet}         % selects Helvetica as sans-serif font
\usepackage{courier}        % selects Courier as typewriter font
\usepackage{type1cm}        % activate if the above 3 fonts are
                            % not available on your system

\usepackage{makeidx}         % allows index generation
\usepackage{graphicx}        % standard LaTeX graphics tool
                             % when including figure files
\usepackage{multicol}        % used for the two-column index
\usepackage[bottom]{footmisc}% places footnotes at page bottom
\usepackage[normalem]{ulem}	% for strike-through of text with \sout{}  
\usepackage{hyperref}  %for hyperlinks

\usepackage{soul}   % for high-lighting of text
% see the list of further useful packages
% in the Reference Guide

% expansions of  journal abbreviations from bibtex entries by ADS
% adapted to Springer Basic style (no periods in abbreviations)
\newcommand*\aap{A\&A}
\let\astap=\aap
\newcommand*\aapr{A\&A Rev}
\newcommand*\aaps{A\&AS}
\newcommand*\actaa{Acta Astron}
\newcommand*\aj{AJ}
\newcommand*\ao{Appl Opt}
\let\applopt\ao
\newcommand*\apj{ApJ}
\newcommand*\apjl{ApJ}
\let\apjlett\apjl
\newcommand*\apjs{ApJS}
\let\apjsupp\apjs
\newcommand*\aplett{Astrophys Lett}
\newcommand*\apspr{Astrophys Space Phys Res}
\newcommand*\apss{Ap\&SS}
\newcommand*\araa{ARA\&A}
\newcommand*\azh{AZh}
\newcommand*\baas{BAAS}
\newcommand*\bac{Bull astr Inst Czechosl}
\newcommand*\bain{Bull Astron Inst Netherlands}
\newcommand*\caa{Chinese Astron Astrophys}
\newcommand*\cjaa{Chinese J Astron Astrophys}
\newcommand*\fcp{Fund Cosmic Phys}
\newcommand*\gca{Geochim Cosmochim Acta}
\newcommand*\grl{Geophys Res Lett}
\newcommand*\iaucirc{IAU Circ}
\newcommand*\icarus{Icarus}
\newcommand*\jcap{J Cosmology Astropart Phys}
\newcommand*\jcp{J Chem Phys}
\newcommand*\jgr{J Geophys Res}
\newcommand*\jqsrt{J Quant Spectr Rad Transf}
\newcommand*\jrasc{JRASC}
\newcommand*\memras{MmRAS}
\newcommand*\memsai{Mem Soc Astron Italiana}
\newcommand*\mnras{MNRAS}
\newcommand*\na{New A}
\newcommand*\nar{New A Rev}
\newcommand*\nat{Nature}
\newcommand*\nphysa{Nucl Phys A}
\newcommand*\pasa{PASA}
\newcommand*\pasj{PASJ}
\newcommand*\pasp{PASP}
\newcommand*\physrep{Phys Rep}
\newcommand*\physscr{Phys Scr}
\newcommand*\planss{Planet Space Sci}
\newcommand*\pra{Phys Rev A}
\newcommand*\prb{Phys Rev B}
\newcommand*\prc{Phys Rev C}
\newcommand*\prd{Phys Rev D}
\newcommand*\pre{Phys Rev E}
\newcommand*\prl{Phys Rev Lett}
\newcommand*\procspie{Proc SPIE}
\newcommand*\qjras{QJRAS}
\newcommand*\rmxaa{Rev Mexicana Astron Astrofis}
\newcommand*\skytel{S\&T}
\newcommand*\solphys{Sol Phys}
\newcommand*\sovast{Soviet Ast}
\newcommand*\ssr{Space Sci Rev}
\newcommand*\zap{ZAp}


\newcommand{\hbindex}[1]{\hl{#1}\index{#1}}  %highlights index entries

\makeindex             % used for the subject index
                       % please use the style svind.ist with
                       % your makeindex program

%%%%%%%%%%%%%%%%%%%%%%%%%%%%%%%%%%%%%%%%%%%%%%%%%%%%%%%%%%%%%%%%%%%%%%%%%%%%%%%%%%%%%%%%%

\begin{document}

\title*{Title: Transit Timing and Duration Variations for the Discovery and Characterization of Exoplanets}
% Use \titlerunning{Short Title} for an abbreviated version of
% your contribution title if the original one is too long
\author{Eric Agol and Daniel C.\ Fabrycky}
% Use 
\authorrunning{Transit timing} % for an abbreviated version of
% your contribution title if the original one is too long
\institute{Eric Agol \at Department of Astronomy, Box 351580, University of Washington, Seattle, WA 98195-1580, USA \email{agol@uw.edu}
\and Daniel C.\ Fabrycky \at Dept.\ of Astronomy \& Astrophysics, University of Chicago, Chicago, IL 60637, USA \email{fabrycky@uchicago.edu}}
%
% Use the package "url.sty" to avoid
% problems with special characters
% used in your e-mail or web address
%
\maketitle


\abstract{Transiting exoplanets in multi-planet systems have non-Keplerian orbits which can cause the times and durations of transits to vary.  We review the theory and observations of transit timing variations (TTV) and transit duration variations (TDV).}

\section{Introduction}

Here we discuss some aspects of planetary orbital physics, to set the stage for TTV and TDV.  Consider the vector stretching from the star of mass $M_\star$ to the planet of mass $M_p$ to be $\mathbf{r}=(x,y,z)$, with a distance $r$ and direction $\mathbf{\hat r}$.  The Keplerian potential, $\phi=-GM/r$ (where $M \equiv M_\star M_p / (M_\star+M_p)$ is called the reduced mass), is one of only two radial, power-law potentials that gives rise to closed orbits \footnote{the other one, the harmonic potential $-kr$ would only have relevance for collisionless orbits within a homogeneous massive body}.  This means that, in the absense of perturbations, there is a strict periodicity $\mathbf{r}(t+P) = \mathbf{r}(t)$.  Moreover, Kepler showed that Tycho Brahe's excellent data for planetary positions were consistent with Copernicus' idea of a heliocentric system only if the planets (including the Earth) followed elliptical paths of semi-major axis $a$, and one focus on the Sun. Newton was successful at finding the principle underlying such orbits, a force law $\mathbf{\ddot{r}}=-G M r^{-2} \mathbf{\hat r}$, which results in a period $P = 2 \pi a^{3/2} G^{-1/2} (M_\star + M_p)^{-1/2}$ (i.e. with the $a$-scaling Kepler found the planets actually obeyed).

This research program was thrown into some doubt by the ``Great Inequality,'' the fact that the orbits of Jupiter and Saturn did not fit the fixed Keplerian ellipse model.  This was overcome by the perturbation theory of Lagrange, which resulted in the first characterization of the masses of those planets (check).  We can recreate the main effect of this insight by writing an additional force to that of gravity of the Sun: 

\begin{equation}
\mathbf{F_{1}} = -G M r_{1}^{-2} \mathbf{\hat r_{1}} - \mathbf{F_{21}},
\end{equation}
where we now specify forces and distances explicitly to planet 1, and add a force of planet 2 on planet 1.  This latter force consists of two terms: 
{\bf I'm not sure the following equation is correct as stands - if $r_1 < r_2$, then first term would be positive}
\begin{equation}
\mathbf{F_{21}} = -G M (r_{1}-r_{2})^{-3} (\mathbf{r_{2}} - \mathbf{r_{1}}) + G M_2 r_{2}^{-2} \mathbf{\hat r_{2}}.
\end{equation}
The first term on the right-hand-side is the direct gravitational acceleration of planet 1 due to planet 2.  The second is a frame-acceleration effect, due to the acceleration the Sun feels due to the second planet.  Since the Sun is fixed at the zero of the frame, this acceleration is manifested by acceleration in the opposite direction of planet 1.  Can we average the force over the 5:2 resonant conjunction timescale, and see what it amounts to for each of the orbital elements? 

Likewise, Leverrier and Adams used the same technique, dynamical perturbations, to discover the first planet by gravitational means. In this case, they did not know the zeroth order solution (i.e. the Keplerian ellipse) for the yet-to-be-discovered Neptune.  In its place, they assumed the Titius-Bode rule held, and sought only the phase of the orbit.  This worked because they only wanted to see how the acceleration, then deceleration, as Uranus passed Neptune, would betray its position on the sky to optical observers.  [Say later: the task that researchers set for themselves to discover planets by TTV is a bit more demanding.  We do not have any hints as to what the planet's orbit might be (neither circular nor roughly obeying some spacing law).  Additionally, the data per orbit is considerably noisy; in only a few cases are the orbit-by-orbit ``chopping'' signal statistically significant after just three transits. Finally, the orbit is only sampled at the transit phase, so opportunities for aliasing of the signal are abundant.]

The discovery of transits marks the first time that data on exoplanets could be precise enough to notice gravitational interactions.  \footnote{Only around the same time (2000) were perturbations noticed in the resonant interaction of the planets of GJ 876.} 

  - Definition of TTVs/TDVs [DF] (Figure? O-C method)
     \citep{2005Sci...307.1288H,2010Sci...330...51H}
  - History: theory, observation  (Schneider TTV, Miralda-Escude' TTV) [EA]
     \citep{2002ApJ...564.1019M,2010Sci...330...51H}


\section{Preliminaries}

Since the gravitational interactions between planets occurs on the orbital timescale, the
amplitude of transit timing variations is proportional to the orbital period of each planet,
as well as a function of other dimensionless quantities.  Thanks to Newton's second law
and Newton's law of gravity, the acceleration of a body does not depend on its own mass.
Thus, the transit timing variations of each planet scale with the masses of the {\it other} bodies
in the system.
In a two-planet system, then, to lowest order in mass ratio,
\begin{eqnarray}
\delta t_1 &\propto& P_1 \frac{m_2}{m_0} f_1(),\cr
\delta t_2 &\propto& P_2 \frac{m_1}{m_0} f_2(),
\end{eqnarray}
where the masses of the star and planets are $m_0, m_1,$ and $m_2$, and $f_1$ and $f_2$
are a function of the period ratio and the angular orbital elements of the planets.


%  - Basic scalings: $\propto$P, $\propto m/M_{star}$ for the perturber, stronger near resonance [EA]
  - Energy/angular momentum conservation [DF]

  - Linear TTV (independently adds from different planets, off resonance)  - 
  - Applications: [EA]
    - Detection
    - Confirmation
    - Characterization
      - Sensitive to density since time-dependent: transits are sensitive to density of star; TTV are sensitive to mass ratio; transit depth radius ratio - so we get density of planet from transits + TTV.   Dimensions of G are density and time.
      - TTV + RV gives Mass + radius ;   CBPs as example  

\section{Theory} % (applied to specific systems - show fits to actual data - either N-body or analytic)
  - TTVs:
    - Inner Keplerian variation;  CBPs as example (Kepler-16) [DF]
    - Near-resonant TTVs - Lithwick et al.  [DF] (Figure - mechanism + data)
       - Degeneracy - multiple resonances can give same solution (Kepler-19); Breaking degeneracy with TDV as well [DF]
    - Chopping/other harmonics - KOI 1353 / KOI-872 [EA] (Figure)
    - Resonance - Kepler-30?  Ne'svorny (1603.07306); Boue'+2012 - Kepler-223 (resonant chain - to fit data \& stability); room for more work on this.  [DF]
    - Exomoons [EA]
    - Light time?  Borkovits deconvolution [DF]
    - Borkovits(?) - KOI 1474 {cleaner example?  Or leave out?  Future -- circumstellar planets in binaries; Schwartz et al. w/ Haghighipour.} [DF]     - TDVs
       - Precession - Kepler-108 {1606.04485} / KOI-142 {Nesvorny} / KOI-13 (Mazeh) - and CBPs turning on or off.  [DF]  Ragozzine/Wolf/Pal/Koscis/Jordan - GR precession - Heyl \& Gladman; J2  (Figure - CBP? - Kepler-47? Kostov? Kepler-35? Try them out.)
    - Exomoons [EA]

\section{Observations/Practical considerations} % [EA]
Confirmation of multi-planet systems in Kepler anti-correlated sinusoids, Ford GPs [DF]
  [ Some firsts to history section; some best-cases as examples in theory section ]
  - Timing precision: [EA]
     - Comes from steepest part of lightcurve ingress/egress
     - Signal-to-noise of TTV/TDV measurements (Carter/Winn; Rogers/Page)
     - Finite-exposure time effects
     - Effects of stellar variability: flux variability, star spots.

\section{Science Results}
    - Best characterization, specifically mass: Kepler-36 - conjunctions/impulse/Hill approximation (N-body) [EA]
    - Other favorite systems? Kepler-11 puffy/packed planets  [DF]
    - Best eccentricity constraint for a super-Earth?  Kepler-36? Include? 
    - Ensemble TTV analysis: Xie - differing architecture for the single-transiters due to less frequent TTV, Hadden-Lithwick - eccentricity distribution; Hot Jupiters lonely (Steffen); Latham     - gas giants less frequent in multi-transiting (no TTVs)  [DF]
    - Measuring masses - Steffen bias?    [DF]
    - N-body modeling of Kepler-systems: Jontof-Hutter  [DF] (Mass-radius Figure? - ask Daniel Jontof-Hutter)  Transparency to avoid big error bars visually dominating.  EA will make the figure.  Referenced Wayne Hu figure on cosmo constraints.
    - CBPs   [DF]

\section{Future} %  [Might have to chop, or at least briefly cover.  Assign at that time!]
  - More thorough TTV analysis: GPs - for measuring transit times
  - Follow-up of Kepler targets
  - Comparison of TTV masses with RV masses:  better constraints
    and confidence in both methods?
  - MCMC with high-multiplicity systems
  - TESS, JWST, CHEOPS, PLATO, ?
  - TTV/TDV of exomoons
  - HZ exoplanets
  - Smaller CBPs
  - Stellar/planet characterization: TTV + RV

References:

Borucki \& Summers
Struve
Dobrovolskis \& Borucki (1996) - DPS, AAS
Miralda-Escude
Schneider
Cabrera
Charbonneau 2000

Neptune:
Bouvard/Adams/Le Verrier/Galle

%% For figures use
%\begin{figure}
%% Use the relevant command for your figure-insertion program
%% to insert the figure file.
%% For example, with the graphicx style use
%\includegraphics[scale=.65]{template_fig1}
%%
%\caption{Example figure. You do not have to worry on the layout as this will be revised by Springer}
%\label{fig:1}       % Give a unique label
%\end{figure}

%\begin{table}
%\caption{Please write your table caption here.}
%\label{tab:1}       % Give a unique label
%%
%% Follow this input for your own table layout
%%
%\begin{tabular}{p{2cm}p{2.4cm}p{2cm}p{4.9cm}}
%\hline\noalign{\smallskip}
%Classes & Subclass & Length & Action Mechanism  \\
%\noalign{\smallskip}\svhline\noalign{\smallskip}
%Translation & mRNA$^a$  & 22 (19--25) & Translation repression, mRNA cleavage\\
%Translation & mRNA cleavage & 21 & mRNA cleavage\\
%Translation & mRNA  & 21--22 & mRNA cleavage\\
%Translation & mRNA  & 24--26 & Histone and DNA Modification\\
%\noalign{\smallskip}\hline\noalign{\smallskip}
%\end{tabular}
%Give details in a table foot note. $^a$ This is a comment to an entry in the table
%\end{table}
%


\begin{acknowledgement}
EA acknowledges support from NASA Grant ...  DCF acknowledges support from...
\end{acknowledgement}

%  IF you do NOT use bibtex, put comments before the following 2 lines
\bibliographystyle{spbasicHBexo}  %for bibtex
\bibliography{agol_fabrycky} %for bibtex-example

\end{document}
